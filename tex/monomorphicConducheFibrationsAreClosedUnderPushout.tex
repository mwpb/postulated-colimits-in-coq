\documentclass{article}
\usepackage[utf8]{inputenc}
\usepackage{calc}
\setcounter{tocdepth}{2}
\providecommand{\keywords}[1]{\textbf{\textit{Keywords---}} #1}
\usepackage{tikz-cd}
\usepackage{tensor}
%\usepackage{hyperref}
% \usepackage{csquotes}
\usepackage{fancyvrb}
\usepackage{verbatim}
\usepackage{amsmath}

\usepackage{amssymb}
\usepackage{amsthm,thmtools}
\usepackage{nameref,hyperref}
% \usepackage{cleveref}
\AtBeginDocument{\usepackage{cleveref}}
\declaretheorem[numberwithin=section]{theorem}
\declaretheorem[sibling=theorem]{exercise}
\declaretheorem[sibling=theorem]{example}
\declaretheorem[sibling=theorem]{non-example}
\declaretheorem[sibling=theorem]{lemma}
\declaretheorem[sibling=theorem]{corollary}
\declaretheorem[sibling=theorem]{proposition}
\declaretheorem[style=definition,sibling=theorem]{definition}
\declaretheorem[style=definition,sibling=theorem]{axiom}
\declaretheorem[style=definition,sibling=theorem]{notation}
\declaretheorem[sibling=theorem]{question}
\declaretheorem[style=remark,sibling=theorem]{remark}
\usepackage{microtype}
\title{Monomorphic Conduch\'{e} Fibrations are Closed Under Pushout}
\author{Work In Progress}
\begin{document}
\maketitle
\begin{abstract}
	In this paper we prove that monomorphic Conduch\'{e} fibrations are closed under pushout in the category of ordinary categories.
	Monomorphic Conduch\'{e} fibrations are equivalently characterised as those monomorphisms that satisfy a certain pullback condition involving hom objects and therefore the main result of this paper applies to other types of internal categories (for instance to categories internal to a topos).
	Our proof strategy is to formulate a word problem for the arrows in the pushout in terms of arrows in each of the summands and solve this word problem using the elementary characterisation of monomorphic Conduch\'{e} fibrations in terms of factorisations.
\end{abstract}
\tableofcontents

\section{Introduction}

A pullback in the category $Cat$ of ordinary categories (i.e. categories internal to $Set$) can be calculated straightforwardly if one knows the pullback of the underlying sets of objects and the pullback of the sets of arrows for the categories involved.
By contrast the construction of the arrow space of a pushout of ordinary categories is more involved.
In the first instance it necessary to describe which of the objects and arrows in the separate summands are conflated in the pushout category.
However in addition the identification of different objects can create new composite arrows in the pushout category.
This means that we must also figure out which of these extra composites are conflated in the pushout category.

If instead we wanted to calculate a pushout in a category of categories internal to an arbitrary topos there is one further difficulty: it is not in general possible to rely on the principle of the excluded middle.
This means that natural arguments involving case analysis that are available in the category of ordinary categories may not be applied directly.
The main work in this paper is to analyse a particular type of pushout of ordinary categories.
The conditions that define this particular type of pushout can be expressed in terms of pullbacks and the inner hom which means that the main result of this paper does indeed hold for pushouts of categories internal to an arbitrary topos.
Indeed in \cite{intuitionistic-double-negation} we use this extra generality to investigate the properties of a pushout of categories internal to a well-adapted model of synthetic differential geometry.

Now we describe the main definitions and results of this paper.

\begin{notation}
	Let $\beta:\mathbb{A}\rightarrow\mathbb{B}$ be a functor between categories.
	We will use $a$, $a_0$, $a_0'$, $a_1$ etc.. to denote arrows in $\mathbb{A}$ and $b$, $b_0$, $b_0'$, $b_1$ etc.. to denote arrows in $\mathbb{B}$.
\end{notation}

\begin{definition}
	Two factorisations $a_1'a_0'=a=a_1a_0$ of an arrow $a$ in $\mathbb{A}$ are \emph{$\beta$-related} iff there exists an arrow $a_2:cod(a_0)\rightarrow cod(a_0')$ in $\mathbb{A}$ such that $a_2a_0=a_0'$, $a_1'a_2=a_1$ and $\beta(a_2)$ is an identity arrow.
	Let $\sim_{a}$ denote the equivalence relation on the set of factorisations of $a$ that is generated by all the $\beta$-related pairs of factorisations of $a$.
\end{definition}

\begin{definition}
	A functor $\beta: \mathbb{A} \rightarrow \mathbb{B}$ is a \emph{Conduch\'{e} fibration} iff for all arrows $a$ in $\mathbb{A}$ and factorisations $\beta(a)=b_1b_0$ in $\mathbb{B}$ then there exists a factorisation $a=a_1a_0$ in $\mathbb{A}$ such that $\beta(a_1)=b_1$ and $\beta(a_0)=b_0$ and if $a_1a_0$ and $a_1'a_0'$ are factorisations of $a$ satisfying $\beta(a_1)=b_1=\beta(a_1')$ and $\beta(a_0)=b_0=\beta(a_0')$ then the factorisation $a_1a_0$ is equivalent to the factorisation $a_1'a_0'$ under $\sim_a$.
\end{definition}

The following abstract characterisation of Conduch\'{e} fibrations was discovered independently in \cite{MR0190142} and \cite{MR0310033} and is Lemma 6.1 in \cite{MR1245798}.

\begin{definition}
	An object $X$ in a category $\mathcal C$ with finite products is \emph{exponentiable} iff the functor $-\times X:\mathcal C \rightarrow \mathcal C$ has a right adjoint.
\end{definition}

\begin{lemma}
	A functor $\beta:\mathbb{A}\rightarrow \mathbb{B}$ is a Conduch\'{e} fibration iff it is an exponentiable object of the (strict) slice category $Cat/\mathbb{B}$.
\end{lemma}

In this paper we will be exclusively concerned with monomorphic Conduch\'{e} fibrations.
Monomorphic Conduch\'{e} fibrations can be characterised in the following way.

\begin{lemma}\label{characterisation-of-monomorphic-discrete-conduche-fibrations}
	A functor $\beta:\mathbb{A}\rightarrow \mathbb{B}$ is a monomorphic Conduch\'{e} fibration iff it is a monomorphism and
	\begin{equation}\label{eq:monomorphic-conduche-fibration-pullback}
		\begin{tikzcd}
			hom(\mathbf{3},\mathbb{A}) \rar{hom(l,\mathbb{A})} \dar{hom(\mathbf{3},\beta)} & hom(\mathbf{2},\mathbb{A}) \dar{hom(\mathbf{2},\beta)} \\
			hom(\mathbf{3},\mathbb{B}) \rar{hom(l,\mathbb{B})} & hom(\mathbf{2},\mathbb{B})
		\end{tikzcd}
	\end{equation}
	is a pullback in $Set$ where $l:\mathbf{2}\rightarrow \mathbf{3}$ is the functor that takes $0 \mapsto 0$ and $1 \mapsto 2$.
\end{lemma}

\begin{remark}
	The characterisation of monomorphic Conduch\'{e} fibrations in \Cref{characterisation-of-monomorphic-discrete-conduche-fibrations} still makes sense in the more general case when $\mathbb{A}$ and $\mathbb{B}$ are categories internal to a topos $\mathcal E$, the arrow $\beta$ is an $\mathcal E$-functor and the square \eqref{eq:monomorphic-conduche-fibration-pullback} is a pullback in $\mathcal E$.
\end{remark}

\begin{remark}
	It is immediate that monomorphic Conduch\'{e} fibrations are \emph{discrete Conduch\'{e} fibrations} (see for instance the Introduction of \cite{MR1667312}).
	It is also immediate that the arrow $\beta$ satisfies the natural adaptation of the `two out of three' property of part 1 of Definition 1.1.3 in \cite{MR1650134}, namely that if $b=b_1b_0$ in $\mathbb{B}$ and any two of $b$, $b_0$ and $b_1$ are in $\beta(\mathbb{A})$ then the third is also.
\end{remark}

Now we are in a position to state the main result of this paper.

\begin{theorem}
	If $\beta:\mathbb{A} \rightarrow \mathbb{B}$ is a monomorphic Conduch\'{e} fibration and
	\begin{equation}
		\begin{tikzcd}
			\mathbb{A} \rar{\gamma} \dar{\beta} & \mathbb{C} \dar{v} \\
			\mathbb{B} \rar{w} & \mathbb{P}
		\end{tikzcd}
	\end{equation}
	is a pushout then $v: \mathbb{C} \rightarrow \mathbb{P}$ is a monomorphic Conduch\'{e} fibration.
\end{theorem}

Now we describe the proof strategy that we use in ths paper.

Applications to integration.

\section{The Category of Reduced Words} % (fold)
\label{sub:the_category_of_reduced_words}

In this section we describe the extra composite arrows that arise when forming a pushout along a monomorphic Conduch\'{e} fibration.
In addition we describe which of these extra arrows are conflated in the pushout category.
More precisely we describe an arrow in the pushout category as an equivalence class of words in the arrows of the two summands.

\begin{notation}\label{a-b-and-c}
	In this section $\mathbb{A}$, $\mathbb{B}$ and $\mathbb{C}$ are ordinary categories (i.e categories internal to $Set$) with arrow sets $A$, $B$ and $C$ and object sets $K$, $N$ and $M$ respectively.
	We write $s_A$, $s_B$ and $s_C$ for the source maps and $t_A$, $t_B$ and $t_C$ for the target  maps of $\mathbb{A}$, $\mathbb{B}$ and $\mathbb{C}$ respectively.
	The arrow $\beta:\mathbb{A}\rightarrow \mathbb{B}$ is a monomorphic Conduch\'{e} fibration (see \Cref{characterisation-of-monomorphic-discrete-conduche-fibrations}).
	The arrow $\gamma: \mathbb{A}\rightarrow \mathbb{C}$ is an arbitrary functor.
\end{notation}

\begin{remark}
	The assumption that $\beta$ is a monomorphism simplifies our task by restricting the number of new composites in the pushout category.
	The assumption that $\beta$ is a Conduch\'e fibration simplifies our description of the composition in the category of reduced words.
	In the following arguments we use case analysis in several places and therefore are also using the principle of the excluded middle (which holds in the boolean topos $Set$).
\end{remark}

\begin{definition}
\label{amalgamated-source-and-target}
	The \emph{amalgamated source and target maps $s$ and $t$} are the arrows $(\iota_0 s_C,\iota_1 s_B),(\iota_0 t_C,\iota_1 t_B):C\amalg (B\setminus \beta(A)) \rightarrow M+_{K}N$ respectively.
\end{definition}

As a first step towards constructing the arrow space of the pushout category we describe a set of words with a composition and chosen identity arrows.
At this stage we do not describe the pushout category in full as we have yet to specify the appropriate equivalence relation.
The following definition can be found in Section I.1 of \cite{MR939612}.

\begin{definition}
	A \emph{deductive system} is a reflexive graph
	\begin{equation*}
		\begin{tikzcd}
			A \rar[yshift=0.5em]{s} \rar[yshift=-0.5em][swap]{t} & O \arrow{l}[description]{e}
		\end{tikzcd}
	\end{equation*}
	and an arrow $\mu:A\tensor[_t]{\times}{_s} A\rightarrow A$ such that $s\mu = s\pi_0$ and $t\mu = t\pi_1$.
\end{definition}

\begin{definition}\label{deductive-system-of-words}
	The \emph{deductive system $\overline{\mathbb{W}}(\mathbb{B},\mathbb{C})$ of words in $\mathbb{B}$ and $\mathbb{C}$} has as objects the set $M+_K N$ and as arrows the set of sequences $(L_0,L_1,...,L_n)$ of elements $L_i\in C\sqcup (B\setminus \beta(A))$ that satisfy
	\begin{align*}
		t(L_{i})&=s(L_{i+1})\\
		L_i \in C&\implies L_{i+1}\in B\setminus \beta(A)\\
		L_i,L_{i+1}\in B\setminus \beta(A)&\implies t_B(L_{i}) \neq s_B(L_{i+1})
	\end{align*}
	where $s$ and $t$ are the amalgamated source and target maps defined in \Cref{amalgamated-source-and-target}.
	The source map $s_W$ of $\mathbb{W}$ is given by $s_W(L_0,L_1,..,L_n)=s(L_0)$, the target map $t_W$ of $\mathbb{W}$ is given by $t_W(L_0,L_1,...,L_n)=t(L_n)$ and the identity map of $\mathbb{W}$ is given by $e_W(m)=(e_C(m))$ where $e_C$ is the identity map of $\mathbb{C}$.
	The composition $\circ_W$ in $\mathbb{W}$ is given by
	\begin{align*}
		&(L_0,L_1,...,L_n)\circ_W (L'_0,L_1',...,L_m')=\\
	 	&\begin{cases}
	 		(L_0,L_1,...,L_n\circ_B L_0',L_1',...,L_m') \text{ if $L_n,L_0'\in B\setminus \beta(A)$ and $s_B(L_n)=t_B(L_0')$}\\
	 		(L_0,L_1,...,L_n\circ_C L_0',L_1',...,L_m') \text{ if $L_n,L_0'\in C$}\\
	 		(L_0,L_1,...,L_n,L_0',L_1',...,L_m') \text{ otherwise}
	 	\end{cases}
	 \end{align*}
	 where $\circ_B$ and $\circ_C$ denote the compositions in $\mathbb{B}$ and $\mathbb{C}$ respectively.
	 Note that in the first of the cases $L_n\circ_B L_0'\in B \setminus \beta(A)$ because $\beta$ is closed under decomposition and that the associativity of $\circ_{\mathbb{W}}$ is inherited from the associativity of $\circ_B$, $\circ_C$ and the concatenation operation.
\end{definition}

\begin{definition}\label{reduced-words-ordinary}
	The \emph{deductive system $\mathbb{W}=\mathbb{W}(\mathbb{B},\mathbb{C})$ of  reduced words in $\mathbb{B}$ and $\mathbb{C}$} has as objects the set $M+_K N$ and as arrows the set
	$$\overline{\mathbb{W}}(\mathbb{B},\mathbb{C})^{\mathbf{2}}/\sim$$
	of equivalence classes of arrows in $\overline{\mathbb{W}}(\mathbb{B},\mathbb{C})$ where the equivalence relation $\sim$ is generated by equivalences of the form:
	\begin{align*}
		(L_0,...,L_i,e_C(m),L_{i+2},...,L_n)&\sim (L_0,...,L_i\circ_B L_{i+2},...,L_n)\text{ if }s_B(L_i)=t_B(L_{i+2})\\
		(L_0,...,L_i,e_C(m),L_{i+2},...,L_n)&\sim (L_0,...,L_i, L_{i+2},...,L_n)\text{ if }s_B(L_i)\neq t_B(L_{i+2})\\
		(L_0,...,L_i,e_C(m))&\sim (L_0,...,L_i)\\
		(e_C(m),L_{i+2},...,L_n)&\sim (L_{i+2},...,L_n)
	\end{align*}
	where $L_i, L_{i+2}\in B\setminus \beta(A)$ by construction.
\end{definition}

\begin{lemma}
	The deductive system $\mathbb{W}$ is a category.
	\begin{proof}
		We need to check that the identity axioms hold.
		If $L_i\in C$ then
		$$(L_0,...,L_i)\circ_W(e_C(m))=(L_0,...,L_i\circ_C e_C(m))=(L_0,...,L_i)$$
		and if $L_i\in B\setminus \beta(A)$ then
		$$(L_0,...,L_i)\circ_W(e_C(m))\sim(L_0,...,L_i)$$
		by definition of $\sim$ in \Cref{reduced-words-ordinary}.
		If $L_0\in C$ then
		$$(e_C(m))\circ_W(L_0,...,L_n) = (e_C(m)\circ_C L_0,...,L_n)=(L_0,...,L_n)$$
		and if $L_0\in B\setminus \beta(A)$ then
		$$(e_C(m))\circ_W(L_0,...,L_n)\sim (L_0,...,L_n)$$
		by definition of $\sim$ in \Cref{reduced-words-ordinary}.
	\end{proof}
\end{lemma}

\begin{definition}\label{category-of-reduced-words}
	The \emph{category $\mathbb{P}$ of words in $\mathbb{B}$ and $\mathbb{C}$ reduced via $\beta$ and $\gamma$} has as objects the set $M$ and as arrows the set
	$${\mathbb{W}}(\mathbb{B},\mathbb{C})^{\mathbf{2}}/\approx$$
	of equivalence classes of arrows in ${\mathbb{W}}(\mathbb{B},\mathbb{C})$ where the equivalence relation $\approx$ is generated by equivalences of the following form.
	If
	$$L_i=L_i'\nu(a_0)\text{ and }L_{i+1}=\eta(a_1)L_{i+1}'$$
	for some $a_0,a_1\in A$, $L_1,L_{i+1}\in C\sqcup (B\setminus \beta(A))$ and $\nu\neq \eta\in \{\beta,\gamma\}$ then
	$$(L_0,...,L'_i\nu(a_0 a_1),L_{i+1}',...,L_n)\approx (L_0,...,L_i,L_{i+1},...,L_n)$$
	and
	$$(L_0,...,L_i,L_{i+1},...,L_n)\approx (L_0,...,L'_i,\eta(a_0 a_1)L_{i+1}',...,L_n)$$
	which is well defined because $B\setminus \beta(A)$ is closed under decomposition.
\end{definition}

\begin{remark}
	The equivalence relation $\approx$ does not identify sequences of the different lengths.
\end{remark}

\section{Solving the Word Problem} % (fold)
\label{sub:solving_the_word_problem}

\begin{notation}
	Let $\mathbb{A}$, $\mathbb{B}$, $\mathbb{C}$, $\beta$ and $\gamma$ be as in \Cref{a-b-and-c} and $s$ and $t$ as in \Cref{amalgamated-source-and-target}.
	Let $\mathbb{P}$ be the category constructed in \Cref{category-of-reduced-words}.
\end{notation}

\begin{definition}
	The functor $\iota_C:\mathbb{C}\rightarrow \mathbb{P}$ is defined by $c\mapsto (c)$.
	We check that
	\begin{equation*}
		c'\circ_C c\mapsto (c'\circ_C c)=(c')\circ_W (c)
	\end{equation*}
	and
	\begin{equation*}
		e_C(m)\mapsto (e_C(m))=e_W(m)
	\end{equation*}
	as required.
\end{definition}

\begin{definition}
	The functor $\iota_B:\mathbb{B}\rightarrow \mathbb{P}$ is defined by
	\begin{equation*}
		b\mapsto \begin{cases}
			(b) \text{ if } b\in B\setminus\beta(A)\\
			(\gamma(a)) \text{ if } \exists a\in A.~b=\beta(a)
		\end{cases}
	\end{equation*}
	we check the following equivalences:
	\begin{itemize}
		\item If $b'=\beta(a')$ and $b=\beta(a)$ then $b'\circ_B b=\beta(a'\circ_A a)$.
		Therefore
		\begin{equation*}
			\iota_B(b'\circ_B b)= (\gamma(a'\circ_A a)) = (\gamma(a')\circ_C\gamma(a)) = (\gamma(a'))\circ_W (\gamma(a)).
		\end{equation*}
		\item If $b'\in B\setminus \beta(A)$ and $b=\beta(a)$ then $b'\circ_B b\in B\setminus\beta(A)$ because $\beta$ is closed under decomposition.
		Therefore
		\begin{align*}
			\iota_B(b'\circ_B b) &= (b'\circ_B b) = (b'\circ_B \beta(a)) \sim (b'\circ_B\beta(a),e_C(s(b))) \\
			&\approx (b',\gamma(a)\circ_C e_C(s(b)) = (b',\gamma(a)) = (b')\circ_W(\gamma(a)).
		\end{align*}
		\item If $b'=\beta(a)$ and $b\in B\setminus\beta(A)$ then $b'\circ_B b\in B\setminus \beta(A)$ because $\beta$ is closed under decomposition.
		Therefore
		\begin{align*}
			\iota_B(b'\circ_B b) &= (b'\circ_B b) = (\beta(a')\circ_B b) \sim (e_C(t(b')),\beta(a')\circ_B b) \\
			&\approx (e_C(t(b'))\circ_C\gamma(a'),b) =(\gamma(a'),b)=(\gamma(a'))\circ_W (b).
		\end{align*}
		\item If $b',b\in B\setminus\beta(A)$ then
		\begin{equation*}
			\iota_B(b'\circ_B b) = (b'\circ_B b) = (b')\circ_W (b).
		\end{equation*}
		\item Finally
		$$\iota_B(e_B(n))=\iota_B(\beta(e_A(n))=(\gamma(e_A(n)))=(e_C(\gamma(n)))=e_W(\gamma(n)).$$
	\end{itemize}
\end{definition}

\begin{lemma}
	The square
	\begin{equation*}
		\begin{tikzcd}
			\mathbb{A} \rar{\gamma} \dar{\beta} &\mathbb{C} \dar{\iota_C}\\
			\mathbb{B} \rar{\iota_B} & \mathbb{P}
		\end{tikzcd}
	\end{equation*}
	is a pushout in $Cat$.
	\begin{proof}
		Let $x:\mathbb{B}\rightarrow \mathbb{X}$ and $y:\mathbb{C}\rightarrow \mathbb{X}$ be functors such that $x\beta = y\gamma$.
		Then we define a functor $z:\mathbb{P}\rightarrow \mathbb{X}$ by
		\begin{equation*}
			(L_0,L_1,...,L_n)\mapsto \theta_0(L_0)\circ_X\theta_1(L_1)\circ_X ...\circ_X\theta_n(L_n)
		\end{equation*}
		where $\theta_i = x$ if $L_i\in B\setminus \beta(A)$ and $\theta_i=y$ if $L_i\in C$.
		First $z\iota_B=x$ because
		\begin{equation*}
			z\iota_B(b) =
			\begin{cases}
				x(b) \text{ if } b\in B\setminus \beta(A)\\
				y(\gamma (a)) = x(\beta(a)) = x(b) \text{ if } \exists a\in A.~\beta(a)=b
			\end{cases}
		\end{equation*}
		and $z\iota_C = y$ is immediate by construction.
		Next we check that $z$ respects the equivalence relation $\sim$.
		\begin{align*}
			&z(L_0,...,L_i,e_C(m),L_{i+1},...,L_N)=\\
			&\theta_0(L_0)\circ_X...\circ_X\theta_i(L_i)\circ_X y(e_C(m)) \circ_X v_{i+2}(L_{i+2}) \circ_X ... \circ_X \theta_n(L_n)=\\
			&\theta_0(L_0)\circ_X...\circ_X\theta_i(L_i)\circ_X v_{i+2}(L_{i+2}) \circ_X ... \circ_X \theta_n(L_n)=\\
			&\begin{cases}
				z(L_0,..,L_i\circ_B L_{i+2},..,L_n)\text{ if }s_B(L_i)=t_B(L_{i+2})\\
				z(L_0,..,L_i, L_{i+2},..,L_n)\text{ if }s_B(L_i)\neq t_B(L_{i+2})
			\end{cases}
		\end{align*}
		where $L_i,L_{i+2}\in B\setminus \beta(A)$ by construction.
		Also
		\begin{align*}
			z(L_0,...,L_i,e_C(m)) &= \theta_0(L_0)\circ_X ... \circ_X\theta_i(L_i)\circ_X y(e_C(m))\\
			&=\theta_0(L_0)\circ_X ... \circ_X\theta_i(L_i)\\
			&=z(L_0,...,L_n)
		\end{align*}
		and
		\begin{align*}
			z(e_C(m),L_{i+2},...,L_n) & = y(e_C(m))\circ_X \theta_{i+2}(L_{i+2})\circ_X ... \circ_X\theta_n(L_n)\\
			&=\theta_{i+2}(L_{i+2})\circ_X ... \circ_X \theta_n(L_n)\\
			&=z(L_{i+2},...,L_n)
		\end{align*}\
		In addition $z$ respects the equivalence relation $\approx$.
		Indeed if $L_i=L_i'\nu(a_0)$ and $L_{i+1}=\eta(a_1)L_{i+1}'$ for some $a_0,a_1\in A$ and $\eta\neq \nu\in\{\beta,\gamma\}$ then
		\begin{align*}
			&z(L_0,...,L_i'\nu(a_0 a_1),L_{i+1}',...,L_n)\\
			&=\theta_0(L_0)\circ_X ... \circ_X \theta_i(L_i'\nu(a_0 a_1))\circ_X\nu_{i+1}(L_{i+1}')\circ_X...\circ_X\theta_n(L_n)\\
			&=\theta_0(L_0)\circ_X ... \circ_X \theta_{i}(L_i')\circ_X \theta_i(\nu(a_0)) \circ_X \theta_i(\nu(a_1)) \circ_X\theta_{i+1}(L_{i+1}')\circ_X ... \circ_X\theta_n(L_n)\\
			&=\theta_0(L_0)\circ_X ... \circ_X \theta_{i}(L_i')\circ_X \theta_i(\nu(a_0)) \circ_X \theta_{i+1}(\eta(a_1)) \circ_X\theta_{i+1}(L_{i+1}')\circ_X ... \circ_X\theta_n(L_n)\\
			&=\theta_0(L_0)\circ_X ... \circ_X \theta_{i}(L_i'\nu(a_0)) \circ_X \theta_{i+1}(\eta(a_1)L_{i+1}')\circ_X ... \circ_X\theta_n(L_n)\\
			&=z(L_0,...,L_i,L_{i+1},...,L_n)
		\end{align*}
		and similarly $z(L_0,...,L_i,L_{i+1},...,L_n)=z(L_0,...,L_i',\eta(a_0 a_1)L_{i+1}',...,L_n)$.
		Now we show that $z$ is the unique map $\mathbb{P}\rightarrow \mathbb{X}$ such that $z\iota_B=x$ and $z\iota_c = y$.
		So suppose that there were another map $w:\mathbb{P}\rightarrow \mathbb{X}$ such that $w\iota_B = x$ and $w\iota_C = y$.
		Then
		\begin{align*}
			w(L_0,...,L_n) &= w(L_0)\circ_X ...\circ_X w(L_n)\\
			&=\theta_0(L_0) \circ_X...\circ_X\theta_n(L_n)\\
			&=z(L_0,...,L_n)
		\end{align*}
		as required.
	\end{proof}
\end{lemma}

\section{Pushout is a Monomorphic Conduch\'{e} Fibration}

\begin{notation}
	We use the symbol $\simeq$ to denote the equivalence relation generated by both $\sim$ and $\approx$.
	As usual whenever we write $L=(L_0,L_1,...,L_n)$ we assume that $L$ is an arrow of the deductive system $\overline{\mathbb{W}}(\mathbb{B},\mathbb{C})$ defined in \Cref{deductive-system-of-words}.
	$A$, $B$, $C$ etc..
\end{notation}

\begin{lemma}\label{all-in-C}
	If $c\in C$ and $(L_0,L_1,...,L_n)\simeq (c)$ then $\forall i\in \{0,...,n\}.~L_i\in C$.
	\begin{proof}
		First note that $\approx$ does not affect the number of $L_i$ that are in $B\setminus \beta(A)$.
		So we only need to consider the case that $(L_0,L_1,...,L_n)\sim (c)$.
		Now consider the four generating relations
		\begin{align*}
			(L_0,...,L_i,e_C(m),L_{i+2},...,L_n)&\sim (L_0,...,L_i\circ_B L_{i+2},...,L_n)\text{ if }s_B(L_i)=t_B(L_{i+2})\\
			(L_0,...,L_i,e_C(m),L_{i+2},...,L_n)&\sim (L_0,...,L_i, L_{i+2},...,L_n)\text{ if }s_B(L_i)\neq t_B(L_{i+2})\\
			(L_0,...,L_i,e_C(m))&\sim (L_0,...,L_i)\\
			(e_C(m),L_{i+2},...,L_n)&\sim (L_{i+2},...,L_n)
		\end{align*}
		for $\sim$ given in \Cref{reduced-words-ordinary}.
		In the first two cases there is at least one $L_i\in B\setminus \beta(A)$ on both sides.
		In the final two cases there is either one $L_i\in B\setminus\beta(A)$ on each side on none on either side.
		The result follows immediately.
	\end{proof}
\end{lemma}

\begin{proposition}\label{pushout-closed-under-decomposition}
	If $\beta$ is closed under decomposition and is the identity on objects and
	\begin{equation*}
		\begin{tikzcd}
			\mathbb{A} \dar{\beta} \rar{\gamma} & \mathbb{C} \dar{\iota_C}\\
			\mathbb{B} \rar{\iota_B} & \mathbb{P}
		\end{tikzcd}
	\end{equation*}
	is a pushout then $\iota_C$ is closed under decomposition.
	\begin{proof}
		We need to prove that if $L=(L_0,L_1,...,L_n)$ and $L'=(L_0',...,L_n')$ such that $L\circ_P L'\simeq (c'')$ for some $c''\in C$ then $L\simeq (c)$ and $L'\simeq (c')$ for some $c,c''\in C$.
		Now consider the definition of $\circ_P$ given in \Cref{deductive-system-of-words}:
		\begin{align*}
		&(L_0,L_1,...,L_n)\circ_W (L'_0,L_1',...,L_m')=\\
	 	&\begin{cases}
	 		(L_0,L_1,...,L_n\circ_B L_0',L_1',...,L_m') \text{ if $L_n,L_0'\in B\setminus \beta(A)$ and $s_B(L_n)=t_B(L_0')$}\\
	 		(L_0,L_1,...,L_n\circ_C L_0',L_1',...,L_m') \text{ if $L_n,L_0'\in C$}\\
	 		(L_0,L_1,...,L_n,L_0',L_1',...,L_m') \text{ otherwise.}
	 	\end{cases}
	 \end{align*}
	 and note that by \Cref{all-in-C} we must be in the second case.
	 Indeed if either of the first or third cases obtained then at least on of $L_n$ and $L_0'$ would be in $B\setminus\beta(A)$ and by Lemma \Cref{all-in-C} this would contradict the fact that $L\simeq (c)$ and $L'\simeq (c')$.

	 So suppose that $L\circ_P L' = (L_0,...,L_n\circ_C L_0',L_1',...,L_m')$.
	 Again \Cref{all-in-C} implies $n,m=0$.
	 Therefore
	 $$L\circ_P L'\simeq (c)\circ_P (c')\simeq (c\circ_C c')\in C$$
	 as required.
	\end{proof}
\end{proposition}

\section{Conclusions}

Deductions, explanations and conclusions.

\bibliography{./references}
\bibliographystyle{plain}

\end{document}
