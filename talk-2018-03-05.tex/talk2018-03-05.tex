\documentclass{article}
\usepackage{calc}
\setcounter{tocdepth}{2}
\providecommand{\keywords}[1]{\textbf{\textit{Keywords---}} #1}
\usepackage{tikz-cd}
\usepackage{tensor}
% \usepackage{hyperref}
% \usepackage{csquotes}
\usepackage{fancyvrb}
\usepackage{todonotes}
\usepackage{verbatim}
\usepackage{amsmath}
\usepackage{microtype}
\usepackage{amssymb}
\usepackage{amsthm}
\usepackage{thmtools}
\usepackage{nameref,hyperref}
\usepackage{cleveref}

\newcommand{\tens}[2]{{_{#1}}\times_{#2}}

\declaretheorem[numberwithin=section]{theorem}
\declaretheorem[sibling=theorem]{exercise}
\declaretheorem[sibling=theorem]{example}
\declaretheorem[sibling=theorem]{non-example}
\declaretheorem[sibling=theorem]{lemma}
\declaretheorem[sibling=theorem]{corollary}
\declaretheorem[sibling=theorem]{proposition}
\declaretheorem[style=definition,sibling=theorem]{definition}
\declaretheorem[style=definition,sibling=theorem]{axiom}
\declaretheorem[style=definition,sibling=theorem]{notation}
\declaretheorem[sibling=theorem]{question}
\declaretheorem[style=remark,sibling=theorem]{remark}
\title{Using Postulated Colimits in Coq}
\author{Work in Progress}
\begin{document}
\maketitle
\begin{abstract}
  In this talk we define and construct finite colimits in the Coq proof assistant in a context that is similar to the category of sets. First we review without proof the key mathematical ideas involved in the theory of postulated colimits as described in a note of Anders Kock. This theory gives us a way to prove results about colimits in an arbitrary sheaf topos. Then we give an inductive definition in Coq of the fundamental notion of zigzag in this theory. We finish by proving the result analogous to the (mathematically easy) result that in the category of sets pushouts of monomorphisms are monomorphisms.
\end{abstract}
\tableofcontents

\section{Postulated Coequalisers}
\label{sec:postulated-coequalisers}

We will use the traditional definition to define a coequaliser.

\begin{definition}\label{def:coequaliser}
  The \emph{coequaliser $q$ of a parallel pair $a,b:R \rightrightarrows X$} is an arrow
    \begin{equation*}
      \begin{tikzcd}
       R \rar[yshift=0.5ex]{s} \rar[yshift=-0.5ex][swap]{t} & X \drar[swap]{\forall z} \rar[twoheadrightarrow]{q} & Q \dar[dashed]{\exists !\psi}\\
        & {} & Z
      \end{tikzcd}
    \end{equation*}
    such that
    \begin{itemize}
    \item the equality $sq=tq$ holds
    \item for all $z:X \rightarrow Z$ such that $sz=tz$ there exists a unique $\psi:Q \rightarrow Z$.
    \end{itemize}
\end{definition}

We will use the idea of a postulated colimit to give an explicit presentation of colimits in the category of sets.
First we modify the definition of zigzag.

\begin{definition}\label{def:zigzag}
  An \emph{$s$-$t$-zigzag from $x\in X$ to $y\in X$} is a sequence $r_1,r_2,...,r_n\in R$ such that
  \begin{equation*}
    \begin{tikzcd}[column sep=0.5cm]
      {} & r_1 \dlar[mapsto]{\phi_1} \drar[mapsto]{\phi_2} & {} & r_2 \dlar[mapsto]{\phi_3} \drar[mapsto]{\phi_4}& {} & r_3 \dlar[mapsto]{\phi_5}\drar[mapsto]{\phi_6} &{} &...& & r_n \dlar[mapsto]{\phi_7}\drar[mapsto]{\phi_8} &{} \\
      x & {} & x_1 & {} & x_2 & {} & {} & ...& {}  & {} & y
    \end{tikzcd}
  \end{equation*}
  where the $\phi_i\in\{s,t\}$.
\end{definition}  

\begin{definition}\label{def:postulated-coequaliser}
  A cofork
  \begin{equation*}
    \begin{tikzcd}
     R \rar[yshift=0.5ex]{s} \rar[yshift=-0.5ex][swap]{t} & X \rar{q} & Q
    \end{tikzcd}
  \end{equation*}
  is \emph{a postulated coequaliser} iff
  \begin{itemize}
  \item the arrow $q$ is an epimorphism
  \item for all $x,y\in X$ $(x)q=(y)q$ iff there exists an $s$-$t$-zigzag from $x$ to $y$.
  \end{itemize}
\end{definition}

We can use postulated coequalisers to give a concrete presentation of coequalisers due to the following result of Kock.

\begin{proposition}\label{prop:postulated-colimits-are-colimits}
  In any Grothendieck toposes $\mathcal E$ every postulated coequaliser is a coequaliser.
\end{proposition}

\section{An Inductive Definition of Zigzag}
\label{sec:inductive-definition-zigzag}




\bibliography{references}
\bibliographystyle{hplain}

\end{document}
