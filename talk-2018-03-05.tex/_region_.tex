\message{ !name(talk2018-03-05.tex)}\documentclass{article}
\usepackage{calc}
\setcounter{tocdepth}{2}
\providecommand{\keywords}[1]{\textbf{\textit{Keywords---}} #1}
\usepackage{tikz-cd}
\usepackage{tensor}
% \usepackage{hyperref}
% \usepackage{csquotes}
\usepackage{fancyvrb}
\usepackage{todonotes}
\usepackage{verbatim}
\usepackage{amsmath}
\usepackage{microtype}
\usepackage{amssymb}
\usepackage{amsthm}
\usepackage{thmtools}
\usepackage{nameref,hyperref}
\usepackage{cleveref}

\newcommand{\tens}[2]{{_{#1}}\times_{#2}}

\declaretheorem[numberwithin=section]{theorem}
\declaretheorem[sibling=theorem]{exercise}
\declaretheorem[sibling=theorem]{example}
\declaretheorem[sibling=theorem]{non-example}
\declaretheorem[sibling=theorem]{lemma}
\declaretheorem[sibling=theorem]{corollary}
\declaretheorem[sibling=theorem]{proposition}
\declaretheorem[style=definition,sibling=theorem]{definition}
\declaretheorem[style=definition,sibling=theorem]{axiom}
\declaretheorem[style=definition,sibling=theorem]{notation}
\declaretheorem[sibling=theorem]{question}
\declaretheorem[style=remark,sibling=theorem]{remark}
\title{Using Postulated Colimits in Coq}
\author{Work in Progress}
\begin{document}

\message{ !name(talk2018-03-05.tex) !offset(-3) }

\maketitle
\begin{abstract}
  In this talk we will define and construct finite colimits in Coq in a context that is ostensibly similar to the category of sets.
  First we will review without proof the key mathematical ideas involved in the theory of postulated colimits as described in a note of Anders Kock.
  Then we will give an inductive definition of the fundamental notion of zigzag in this theory.
\end{abstract}
\tableofcontents



\bibliography{references}
\bibliographystyle{hplain}

\message{ !name(talk2018-03-05.tex) !offset(-5) }

\end{document}
