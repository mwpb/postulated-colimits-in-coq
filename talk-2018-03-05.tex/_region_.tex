\message{ !name(talk2018-03-05.tex)}\documentclass{article}
\usepackage{calc}
\setcounter{tocdepth}{2}
\providecommand{\keywords}[1]{\textbf{\textit{Keywords---}} #1}
\usepackage{tikz-cd}
\usepackage{tensor}
% \usepackage{hyperref}
% \usepackage{csquotes}
\usepackage{fancyvrb}
\usepackage{todonotes}
\usepackage{verbatim}
\usepackage{amsmath}
\usepackage{microtype}
\usepackage{amssymb}
\usepackage{amsthm}
\usepackage{thmtools}
\usepackage{nameref,hyperref}
\usepackage{cleveref}
\usepackage[utf8]{inputenc}

\newcommand{\tens}[2]{{_{#1}}\times_{#2}}

\declaretheorem[numberwithin=section]{theorem}
\declaretheorem[sibling=theorem]{exercise}
\declaretheorem[sibling=theorem]{example}
\declaretheorem[sibling=theorem]{non-example}
\declaretheorem[sibling=theorem]{lemma}
\declaretheorem[sibling=theorem]{corollary}
\declaretheorem[sibling=theorem]{proposition}
\declaretheorem[style=definition,sibling=theorem]{definition}
\declaretheorem[style=definition,sibling=theorem]{axiom}
\declaretheorem[style=definition,sibling=theorem]{notation}
\declaretheorem[sibling=theorem]{question}
\declaretheorem[style=remark,sibling=theorem]{remark}
\title{Using Postulated Colimits in Coq}
\author{Work in Progress}
\begin{document}

\message{ !name(talk2018-03-05.tex) !offset(199) }
\section{Pushouts of Monomorphisms in a Topos are Monomorphisms}
\label{sec:pushouts-of-monomorphisms-are-monomorphisms}

In a general topos we can make the following argument.
For simplicity we assume that we are working in the category of sets and functions.

\begin{lemma}\label{lem:pushout-of-mono}
  If $f$ is a monomorphism and
  \begin{equation*}
    \begin{tikzcd}
     A \rar{g} \dar[rightarrowtail]{f} & C\dar{i_1}\\
     B \rar{i_0} & Q
    \end{tikzcd}
  \end{equation*}
  is a pushout in $Set$ then $i_1$ is a monomorphism.
  \begin{proof}
    We first note that $(f,g):A \rightarrowtail B\times C$ is a monomorphism because $f$ is a monomorphism.
    Next
    \begin{equation*}
      \begin{tikzcd}
       A \rar{g} \dar[rightarrowtail]{f} & C\dar{i_1} \arrow[bend left,rightarrowtail]{ddr}{c\mapsto \{c\}} & {}\\
       B \rar{i_0} \arrow[bend right]{rrd}{\phi_{(f,g)}} & Q \drar[dashed]{\psi} & {}\\
       {} & {} & \mathcal P C
      \end{tikzcd}
    \end{equation*}
    where $\phi_{(f,g)}(b)=\{c\in C|~\exists a\in A.~f(a)=b \wedge g(a)=c\}$.
    Therefore $i_1$ is a monomorphism because $c\mapsto \{c\}$ is a monomorphism.
  \end{proof}
\end{lemma}

The proof that we will generalise is more straightforward.
We sketch the argument below.

\begin{lemma}\label{lem:pushout-of-mono-2}
  If $f$ is a monomorphism and
  \begin{equation*}
    \begin{tikzcd}
     A \rar{g} \dar[rightarrowtail]{f} & C\dar{i_1}\\
     B \rar{i_0} & Q
    \end{tikzcd}
  \end{equation*}
  is a pushout in $Set$ then $i_1$ is a monomorphism.
  \begin{proof}

  \end{proof}
  Suppose that $c1,c2\in C$ such that $i_1(c1)\sim i_1(c2)$.
  This means that we can find a zigzag
  \begin{equation*}
    \begin{tikzcd}[column sep=0.3cm]
      {} & a_1 \dlar[mapsto]{g} \drar[mapsto]{f} & {} & a_2 \dlar[mapsto]{f} \drar[mapsto]{g}& {} & a_3 \dlar[mapsto]{g}\drar[mapsto]{f} &{} &...& & a_n \dlar[mapsto]{f}\drar[mapsto]{g} &{} \\
      x_1 & {} & x_2 & {} & x_3 & {} & {} & ...& {}  & {} & x_{n+1}
    \end{tikzcd}
  \end{equation*}
  where $x_1\sim c1$, $x_{n+1}\sim c2$ and the order of the $f$ and $g$s has been forced by the disjoint union.
  (As well as the fact that we are starting and ending at a $g$.)
  However since $f$ is a monomorphism we see that $a_{2i-1}\sim a_{2i}$ and so
  \begin{equation*}
    c1 \sim x_1 \sim x_3 \sim x_5 \sim ... \sim x_{n+1}\sim c2
  \end{equation*}
  as required.
\end{lemma}

This is a test

\begin{equation*}
  e=mc^2
\end{equation*}

\bibliography{references}
\bibliographystyle{hplain}

\message{ !name(talk2018-03-05.tex) !offset(197) }

\end{document}
